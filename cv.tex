% Generated from cv.json via scripts/generate-cv.mjs

\documentclass[11pt,a4paper,sans]{moderncv}

\moderncvcolor{blue}

\moderncvstyle{classic}

\usepackage[scale=0.75]{geometry}

\usepackage[english]{babel}

\name{Adam DJ}{Brett}

\title{International Research Associate}

\address{6339 Charlotte Pike \#532}{Nashville, Tennessee 37029}{US}

\email{info@adamdjbrett.com}

\homepage{https://www.adamdjbrett.com}

\social[linkedin]{adamdjbrett}
\social[orcid]{0009-0004-6725-8425}

\begin{document}

\microtypesetup{expansion=false}

\makecvtitle

\section{Professional Summary}
\cvitem{}{Operations Manager and International Research Associate with the American Indian Law Alliance and a Visiting Professor of Digital Humanities and Indigenous Studies at United Lutheran Seminary}

\section{Experience}

\cventry{Current}{Operations Manager \& International Research Associate}{American Indian Law Alliance}{Remote}{}{Managing operations and conducting international research initiatives with the American Indian Law Alliance, a prominent organization focused on Indigenous rights and advocacy.}
\cventry{Current}{Visiting Professor of Digital Humanities and Indigenous Studies}{United Lutheran Seminary}{Remote}{}{Teaching courses in digital humanities and Indigenous studies, bridging academic research with digital methodologies and Indigenous knowledge systems.}
\cventry{Current}{Part-time Faculty - Department of Religion}{Syracuse University}{Remote}{}{Part-time teaching faculty in the Department of Religion at Syracuse University, contributing to courses in religious studies and related disciplines.}

\section{Education}

\cventry{2008--2015}{PhD in Religion}{Syracuse University}{}{}{Doctoral research focused on Religion in the Americas and Indigenous religions}
\cventry{2002--2006}{Master of Theological Studies (MTS) in Divinity}{Brite Divinity School at Texas Christian University}{}{}{}
\cventry{2004--2007}{Master of Theology (ThM) in Divinity}{Brite Divinity School at Texas Christian University}{}{}{}
\cventry{1997--2002}{Bachelor of Arts (BA) in General Studies}{Warner University}{}{}{}

\section{Skills}

\cvitem{Digital Humanities}{Digital scholarship, Web development, Data visualization, Digital archives, Electronic publishing}
\cvitem{Indigenous Studies}{Indigenous rights, Decolonization, Indigenous history, Traditional knowledge, Indigenous governance}
\cvitem{Operations Management}{Strategic planning, Project management, Organizational coordination, Process improvement, International relations}
\cvitem{Academic Research}{Religious studies, Ethnography, Qualitative research, Literature review, Academic writing}
\cvitem{Web Technologies}{HTML, CSS, JavaScript, Eleventy, Static site generators, Web administration}
\cvitem{Web Administration}{Site maintenance, Server management, CMS administration, Technical support}
\cvitem{Documentation}{Technical writing, API documentation, User guides, Markdown, RST}

\section{Volunteer}
\cvitem{Contributor \& Author}{Doctrine of Discovery Project: Contributing research and commentary on the Doctrine of Discovery and its impacts on Indigenous peoples and international law.}

\section{Languages}
\cvitemwithcomment{English}{Native speaker}{}

\section{Interests}
\cvitem{Indigenous Knowledge Systems}{Traditional ecological knowledge, Indigenous governance, Indigenous pedagogy}
\cvitem{Digital Culture and Technology}{Digital humanities, Technology ethics, Open-source movements}
\cvitem{Religious Studies}{Comparative religion, Religion in the Americas, Theology and social justice}
\cvitem{International Law and Rights}{Indigenous rights, International advocacy, Doctrine of Discovery}

\section{Publications}
\begin{itemize}
\item Examining the Doctrine of Discovery in Religion and Indigenous Studies (Religion Compass, 2025). Co-author with Betty Hill. Analyzes the matrix of enslavement, exploitation, and extraction within settler-colonial systems and examines scholarly engagement with the Doctrine of Discovery across Religious Studies, Legal Studies, and Indigenous Studies.\newline\href{https://doi.org/10.1111/rec3.70039}{https://doi.org/10.1111/rec3.70039}
\item Healing the Sacred: The Fight to Restore Onondaga Lake and Honor Indigenous Land (International Journal on Responsibility, 2025). Co-author with Betty Hill (Lyons). Examines environmental restoration and Indigenous land rights in the context of Onondaga Lake.\newline\href{https://doi.org/10.62365/2576-0955.1133}{https://doi.org/10.62365/2576-0955.1133}
\item Documenting Domination in International Relations Through the Doctrine of Discovery (Cross Currents, 2024). Co-author with Betty Hill. Discusses AILA's work to dismantle the Doctrine of Discovery through international advocacy and collaboration, emphasizing the shift from rights-based to responsibility-based approaches.\newline\href{https://doi.org/10.1353/cro.2024.a963640}{https://doi.org/10.1353/cro.2024.a963640}
\item Book Notes: New Books in the Study of Domination (Cross Currents, 2024). Co-author with Betty Hill. Reviews books on domination, including works on Indigenous values, African Diaspora religions, and systemic oppression.\newline\href{https://doi.org/10.1353/cro.2024.a963642}{https://doi.org/10.1353/cro.2024.a963642}
\item 200 Years of Johnson v. M'Intosh: Indigenous Responses to the Religious Foundations of Racism (Cross Currents, 2024). Co-author with Philip P. Arnold and Sandra Bigtree. Examines the religious foundations of racism through analysis of the Johnson v. M'Intosh decision.\newline\href{https://doi.org/10.1353/cro.2024.a963625}{https://doi.org/10.1353/cro.2024.a963625}
\item The Religious Origins of White Supremacy and The Doctrine of Christian Discovery (Cross Currents, 2024). Co-author with Philip P. Arnold and Sandra Bigtree. Explores the theological and historical foundations of white supremacy through the Doctrine of Christian Discovery.\newline\href{https://doi.org/10.1353/cro.2024.a963641}{https://doi.org/10.1353/cro.2024.a963641}
\item Documenting Domination: From the Doctrine of Christian Discovery to Dominion Theology (Religions, 2023). Co-author with Betty Hill (Lyons). Traces the genealogy of Christian domination from fifteenth-century papal bulls through contemporary Christian nationalist theologies, analyzing colonial charters and primary sources.\newline\href{https://doi.org/10.3390/rel15121493}{https://doi.org/10.3390/rel15121493}
\item On the Limits of the Concept of Religious Freedom in Indigenous Communities (Journal of the Council for Research on Religion, 2024). Co-author with Betty Hill (Lyons). Critiques international legal framings of religious freedom, highlighting limitations in rights-based discourse and offering alternatives based on sovereignty and Two Row Wampum principles.\newline\href{https://doi.org/10.26443/jcreor.v5i2.117}{https://doi.org/10.26443/jcreor.v5i2.117}
\item Preface and Introduction: From Indigenous Religions to Indigenous Values Vol. 5 No. 2 (Journal of the Council for Research on Religion, 2024). Co-author with Philip P. Arnold and Sandra Bigtree. Introduces special issue on Indigenous religious freedom and belief, featuring contributions from Indigenous NGOs, nations, and leaders.\newline\href{https://doi.org/10.26443/jcreor.v5i2.108}{https://doi.org/10.26443/jcreor.v5i2.108}
\item Digital Humanities as Preserving Inherently Ephemeral Information (Bulletin for the Study of Religion, 2023). Explores the role of digital humanities in preserving ephemeral information and knowledge systems, particularly relevant to Indigenous scholarship.\newline\href{http://dx.doi.org/10.1558/bsor.28913}{http://dx.doi.org/10.1558/bsor.28913}
\item However, Extravagant The Pretensions Of Johnson V. M'Intosh (Canopy Forum: On the Interactions of Law and Religion, 2023). Co-author with Betty Hill (Lyons). Peer-reviewed article examining Johnson v. M'Intosh and its legal and religious implications.\newline\href{https://canopyforum.org/2023/03/23/however-extravagant-the-pretensions-of-johnson-v-mintosh/}{https://canopyforum.org/2023/03/23/however-extravagant-the-pretensions-of-johnson-v-mintosh/}
\item Introduction to 200 Years of Johnson v. M'Intosh: Law, Religion, and Native American Lands (Canopy Forum, 2023). Co-author with Philip P. Arnold and Sandra Bigtree. Introduction to thematic series examining 200 years of Johnson v. M'Intosh, its legal foundations, and religious origins.\newline\href{https://canopyforum.org/2023/03/10/introduction-to-the-200-years-of-johnson-v-mintosh-law-religion-and-native-american-lands-series/}{https://canopyforum.org/2023/03/10/introduction-to-the-200-years-of-johnson-v-mintosh-law-religion-and-native-american-lands-series/}
\item Reavers ain't men [sic]--or they forgot how to be: Teaching Religion in Science Fiction (Perspectives in Religious Studies, 2018). Explores pedagogy and methodology for teaching religion through science fiction narratives, examining themes of humanity, identity, and religious meaning.\newline\href{https://libezproxy.syr.edu/login?url=http://search.ebscohost.com/login.aspx?direct=true\&db=rfh\&AN=ATLAi5IE200229000408\&site=ehost-live}{https://libezproxy.syr.edu/login?url=http://search.ebscohost.com/login.aspx?direct=true\&db=rfh\&AN=ATLAi5IE200229000408\&site=ehost-live}
\item Religion and culture: contemporary practices and perspectives (Book Review) (Perspectives in Religious Studies, 2014). Review of contemporary practices and perspectives in the study of religion and culture.
\item The world's religions: a contemporary reader (Book Review) (Perspectives in Religious Studies, 2014). Review of contemporary reader on world religions.
\item Religious and Sexual Nationalisms in Central and Eastern Europe (Book Review) (Journal of Empirical Theology, 2015). Review of 'Religious and Sexual Nationalisms in Central and Eastern Europe: Gods, Gays and Governments' by Sremac and Ganzevoort.\newline\href{https://brill.com/view/journals/jet/29/1/article-p133\_6.xml}{https://brill.com/view/journals/jet/29/1/article-p133\_6.xml}
\item Catastrophic Christianity: An Iconological Study of the Messianic Idea in American Protestant Christianity Circa 1900-1940 (Syracuse University, 2021). PhD Dissertation. Analyzes the messianic idea in America through four case studies examining fundamentalism, authoritarianism, commodification of religion, and hucksterism in early 20th century U.S. culture.\newline\href{https://surface.syr.edu/etd/1544/}{https://surface.syr.edu/etd/1544/}
\end{itemize}

\end{document}